%====================================================================================================
% ?????
%====================================================================================================
% TCC
%----------------------------------------------------------------------------------------------------
% Autor				: Jasane Schio
% Orientador		: Gedson Faria
% Co-Orientador		: Angelo Darcy
% Instituição 		: UFMS - Universidade Federal do Mato Grosso do Sul
% Departamento		: CPCX - Sistema de Informação
%----------------------------------------------------------------------------------------------------
% Data de criação	: 01 de Outubro de 2015
%====================================================================================================

\chapter{Introdução} \label{Cap:Introducao}
Avanços tecnológicos tentam a cada dia mais dar vida as máquinas aplicando sentidos e percepções que se assemelham as humanas. Dentre estes, Inteligência Artificial, Aprendizado de Máquina, processamento de áudio, processamento de imagem entre outros. Segundo Albuquerque\cite{Albuquerque:2001} processar uma imagem, da mesma maneira que o nosso sistema visual humano é capaz de fazer é extremamente complexo, realizar as mesmas tarefas com a ajuda de máquinas, exige por antecedência uma compreensão “filosófica” do mundo ou dos conhecimentos humanos. Sem esse conhecimento pré existente por parte da máquina, a interpretação de uma imagem e seu processamento se baseia nas informações contidas na mesma.
	
	 Essa obtenção e entendimento das informações contidas na imagem se dá pela Visão Computacional, o ramo encarregado por simular o sistema visual humano. O que é feito de uma maneira única pelo sentido humano é separado em varias tarefas dentro da Visão Computacional com a captura de imagem, seu processamento, aquisição de informações da mesma, processamento dessa informação e aplicação de parâmetros para classificação da informação entre outros. Gonzalez\cite{Gonzalez:2008} descreve que uma imagem digital é composta por um número finito de elementos, cada um dos quais tem um determinado local e valor, assim o Processamento de Imagem Digital tem como tarefa a retirada de informações dos elementos de uma imagem.
	
	Dentre tipos de processamento de imagens existem, Gonzalez\cite{Gonzalez:2008} os define como: aplicações de ações primitivas de modificação de imagem, esta caraterizada por seu resultado final ser também uma imagem semelhante a imagem inicial porém modificada (Low-Level-Process), divisão de imagem em regiões e alguns tipos de reconhecimento e classificação de objetos, caraterizada por seu resultado final ser muitas vezes apenas regiões ou informações da imagem inicial (Mid-Level-Process), e o mais “sensorial” de todos que é a analise de objetos usando funções cognitivas associadas a visão computacional, essa usa informações relevantes para o reconhecimento de objetos (Higher-Level-Process).     
%
%	Neste trabalho propõe-se a realização de um processo de reconhecimento do objetos em imagens em tempo real coloridas com tempo de maquina reduzido. Este processamento irá ser realizado em imagens de futebol de robô da categoria Very Small Size, que serão processadas usando modelo de cores HPG e usando intervalos de valores de histograma de cores como parâmetro para detecção do objeto. O reconhecimento desses objetos também apresentara uma forma dinâmica de alocamento de equipes onde o sistema detectara os times e seus participantes de forma autônoma. 
 
	Neste trabalho propõe-se o a automatização do processo de detecção de objetos desenvolvendo um sistema de calibração de intervalo de Mínimos e Máximos dos valores HSV de cores para ser usado pela equipe de Futebol de Robôs Cedro, categoria Very Small Size. Valores HSV são um dos tipos de valores usados para definir as cores em computação, esses valores são correspondentemente Hue, a cor pura, Saturation, o grau de pureza da cor, e Lightness, que é o luminosidade aplicada. No sistema além da detecção automática estará também disponível a detecção manual de objetos. 


\section{Justificativa}
\begin{itemize}

%	\item A ausência de métodos que facilitem a calibração de corem nas bibliotecas de processamento de imagem já existentes
	\item A falta, na equipe, de um sistema de fácil manuseio para detecção de valores HSV
	\item A falta, na equipe, de um sistema automático de registro do valores HSV mínimos e máximos
	\item A falta, na equipe, de um sistema que defina mínimos e máximos de forma automática, baseando-se nos objetos escolhidos
	\item A falta, na equipe, de um sistema autônomo de calibração de intervalo de cores
	\item Aplicação do sistema proposto na identificação de robôs moveis em times de futebol de robôs.
	
			

\end{itemize}
\section{Objetivos} \label{Sec:Objetivos}

\subsection{Objetivo Geral} \label{Sec:ObjetivoGeral}
Este trabalho tem por objetivo principal automatizar o sistema de identificação de objetos 
coloridos em imagens provenientes de uma câmera  em tempo real, fazendo a calibração de intervalo de Mínimos e Máximos dos valores HSV.  
Para alcançar o objetivo principal, foram propostos os seguintes objetivos específicos.

\subsection{Objetivos Específicos} \label{Sec:ObjetivosEspecificos}

\begin{itemize}
	
%	\item Detecção e calculo dos valores mínimos e máximos para cada cor
	\item Implementar uma interface que conte com disposição de informações no estilo gráfico ou histograma de cores para um corte manual de valores
	visando diminuição da velocidade de detecção; 
	\item Estudo e implementação de um sistema inteligente de calibração de cores e no corte inteligente de valores minimo e máximo das cores
	\item Testar o sistema proposto para identificação de equipes e participantes do futebol 
	de robô na categoria Very Small Size. 
	
	
\end{itemize}

\newpage

%\section{Organização da Proposta} \label{Sec:Organizacao}

